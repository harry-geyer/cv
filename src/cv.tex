\documentclass[10pt, letterpaper]{article}

\usepackage[
    ignoreheadfoot,
    top=1.0 cm,
    bottom=0.7 cm,
    left=1.7 cm,
    right=1.7 cm,
    footskip=1.5 cm,
]{geometry}
\usepackage{titlesec}
\usepackage{tabularx}
\usepackage{array}
\usepackage[dvipsnames]{xcolor}
\definecolor{primaryColor}{RGB}{0, 0, 0}
\usepackage{enumitem}
\usepackage{fontawesome5}
\usepackage{amsmath}
\usepackage[
    pdftitle={Harry Geyer's CV},
    pdfauthor={Harry Geyer},
    pdfcreator={Harry Geyer},
    colorlinks=true,
    urlcolor=primaryColor
]{hyperref}
\usepackage[pscoord]{eso-pic}
\usepackage{calc}
\usepackage{bookmark}
\usepackage{lastpage}
\usepackage{changepage}
\usepackage{paracol}
\usepackage{ifthen}
\usepackage{needspace}
\usepackage{iftex}
\usepackage[none]{hyphenat}

\ifPDFTeX
    \input{glyphtounicode}
    \pdfgentounicode=1
    \usepackage[T1]{fontenc}
    \usepackage[utf8]{inputenc}
    \usepackage{lmodern}
\fi

\usepackage{charter}

\raggedright
\AtBeginEnvironment{adjustwidth}{\partopsep0pt}
\pagestyle{empty}
\setcounter{secnumdepth}{0}
\setlength{\parindent}{0pt}
\setlength{\topskip}{0pt}
\setlength{\columnsep}{0.15cm}
\pagenumbering{gobble}

\titleformat{\section}{\needspace{4\baselineskip}\bfseries\large}{}{0pt}{}[\vspace{1pt}\titlerule]

\titlespacing{\section}{-1pt}{0.3 cm}{0.2 cm}

\renewcommand\labelitemi{$\vcenter{\hbox{\small$\bullet$}}$}
\newenvironment{highlights}{
    \begin{itemize}[
        topsep=0.15 cm,
        parsep=0.15 cm,
        partopsep=0pt,
        itemsep=0pt,
        leftmargin=0 cm + 10pt
    ]
}{
    \end{itemize}
}


\newenvironment{highlightsforbulletentries}{
    \begin{itemize}[
        topsep=0.10 cm,
        parsep=0.10 cm,
        partopsep=0pt,
        itemsep=0pt,
        leftmargin=10pt
    ]
}{
    \end{itemize}
}

\newenvironment{onecolentry}{
    \begin{adjustwidth}{
        0 cm + 0.00001 cm
    }{
        0 cm + 0.00001 cm
    }
}{
    \end{adjustwidth}
}

\newenvironment{twocolentry}[2][]{
    \onecolentry
    \def\secondColumn{#2}
    \setcolumnwidth{\fill, 4.5 cm}
    \begin{paracol}{2}
}{
    \switchcolumn \raggedleft \secondColumn
    \end{paracol}
    \endonecolentry
}

\newenvironment{threecolentry}[3][]{
    \onecolentry
    \def\thirdColumn{#3}
    \setcolumnwidth{, \fill, 4.5 cm}
    \begin{paracol}{3}
    {\raggedright #2} \switchcolumn
}{
    \switchcolumn \raggedleft \thirdColumn
    \end{paracol}
    \endonecolentry
}

\newenvironment{header}{
    \setlength{\topsep}{0pt}\par\kern\topsep\centering\linespread{1.5}
}{
    \par\kern\topsep
}

\newcommand{\placelastupdatedtext}{
    \AddToShipoutPictureFG*{
    \put(
        \LenToUnit{\paperwidth-2 cm-0 cm+0.05cm},
        \LenToUnit{\paperheight-1.0 cm}
    ){\vtop{{\null}\makebox[0pt][c]{
        \small\color{gray}\textit{Last updated in August 2025}\hspace{\widthof{Last updated in August 2025}}
    }}}
  }
}

\let\hrefWithoutArrow\href

\begin{document}
    \newcommand{\AND}{\unskip
        \cleaders\copy\ANDbox\hskip\wd\ANDbox
        \ignorespaces
    }
    \newsavebox\ANDbox
    \sbox\ANDbox{$|$}


    \begin{header}
        \fontsize{25 pt}{25 pt}\selectfont Harry Geyer

        \vspace{5 pt}

        \normalsize
        \mbox{Newcastle-under-Lyme, Staffordshire, UK}
        \kern 5.0 pt
        \AND
        \kern 5.0 pt
        \mbox{\hrefWithoutArrow{mailto:EMAIL_ADDRESS}{EMAIL_ADDRESS}}
        \kern 5.0 pt
        \AND
        \kern 5.0 pt
        \mbox{\hrefWithoutArrow{tel:PHONE_NUMBER_BASIC}{PHONE_NUMBER_NICE}}
        \kern 5.0 pt
        \AND
        \kern 5.0 pt
        %\mbox{\hrefWithoutArrow{https://WEBSITE_URL/}{WEBSITE_URL}}
        %\kern 5.0 pt
        %\AND
        \kern 5.0 pt
        \mbox{\hrefWithoutArrow{https://linkedin.com/in/harry-geyer}{linkedin.com/in/harry-geyer}}
        \kern 5.0 pt
        \AND
        \kern 5.0 pt
        \mbox{\hrefWithoutArrow{https://github.com/harry-geyer}{github.com/harry-geyer}}
    \end{header}

    \vspace{5 pt - 0.3 cm}

    \section{Experience}
        \begin{twocolentry}{
            2022 – Present
        }
            \textbf{Firmware Engineer}, Devtank Ltd. -- Derby, UK
        \end{twocolentry}

        \vspace{0.10 cm}
        \begin{onecolentry}
            \begin{highlights}
                \item Sole firmware engineer responsible for embedded development across Devtank’s industrial IoT product line, deployed by multiple industrial clients.
                \item Designed and implemented real-time firmware on STM32/ESP32 MCUs, optimizing statistical calculations (mean, variance, skew) to run in <10 clock cycles, enabling high-frequency sensor processing.
                \item Upstreamed patch to the Linux kernel I2C driver, expanding functionality adopted in production hardware.
                \item Defined embedded architecture and reviewed codebases for reliability, ensuring 0 critical defects in shipped releases.
                \item Partnered with customers on-site to debug and integrate hardware, reducing deployment time by 30%%.
            \end{highlights}
        \end{onecolentry}


        \vspace{0.2 cm}

        \begin{twocolentry}{
            2020 – 2022
        }
            \textbf{Software Development Engineer}, Devtank Ltd. -- Derby, UK
        \end{twocolentry}

        \vspace{0.10 cm}
        \begin{onecolentry}
            \begin{highlights}
                \item Co-architected scalable backend services for monitoring/managing IoT devices, supporting 24/7 uptime for industrial clients with zero downtime tolerance.
                \item Automated internal workflows with Python and Bash, saving ~5 hours/week of manual engineering effort.
                \item Delivered contracted technical reports explaining complex scientific results, tailored for both technical and non-technical readers.
                \item Collaborated directly with clients, delivering solutions on-time and on-budget while ensuring production continuity.
            \end{highlights}
        \end{onecolentry}

    \section{Education}
        \begin{twocolentry}{
            2017 – 2020
        }
            \textbf{University of Warwick}, B.Sc. (Hons) in Physics. 2.1, Second Class, Upper Division.
        \end{twocolentry}

        \vspace{0.10 cm}
        \begin{onecolentry}
            \begin{highlights}
                \item Modules include: computational physics, scientific computing, fluid dynamics, electrical power generation.
                \item Final project: data analysis \& modeling of astronomical dust extinction.
            \end{highlights}
        \end{onecolentry}

        \begin{twocolentry}{
            2015 – 2017
        }
            \textbf{A-Levels}, 4 A to B
        \end{twocolentry}

        \begin{twocolentry}{
            2010 – 2015
        }
            \textbf{GCSEs}, 11 A* to B
        \end{twocolentry}


    \section{Technical Skills}

        \begin{onecolentry}
            \textbf{Languages}: C, C++, Python, SQL, Make, bash (advanced); Rust, JavaScript (ECMAScript), TypeScript, GLSL (proficient).
        \end{onecolentry}

        \vspace{0.2 cm}

        \begin{onecolentry}
            \textbf{Embedded}: STM32, ESP32, libopencm3, ESP-IDF, I2C/SPI/UART, RTOS concepts, Linux kernel driver dev, JTAG/GDB debugging.
        \end{onecolentry}

        \vspace{0.2 cm}

        \begin{onecolentry}
            \textbf{Software}: Flask, Django, PostgreSQL, MySQL, CI/CD pipelines.
        \end{onecolentry}

        \vspace{0.2 cm}

        \begin{onecolentry}
            \textbf{Tools}: Git, pytest, emscripten, OpenGL/WebGL, Docker.
        \end{onecolentry}


    \section{Projects \textit(Links provided)}

        \begin{twocolentry}{\href{https://github.com/harry-geyer/eese}{github.com/harry-geyer/eese}}
            \textbf{Embedded Environmental Sensor Example}
        \end{twocolentry}

        \vspace{0.10 cm}
        \begin{onecolentry}
            Firmware for IoT sensor (C, libopencm3, gdb, Python, pytest). Implemented custom packet detection and corruption identification with minimal overhead.
        \end{onecolentry}

        \vspace{0.2 cm}

        \begin{twocolentry}{\href{https://github.com/harry-geyer/pgwb}{github.com/harry-geyer/pgwb}}
            \textbf{Procedurally Generated Web Background}
        \end{twocolentry}

        \vspace{0.10 cm}
        \begin{onecolentry}
            OpenGL/WebGL rendering engine compiled to WebAssembly (C, OpenGL, WebAssembly, emscripten, JavaScript, HTML, Make).
        \end{onecolentry}

        \vspace{0.2 cm}

        %\begin{twocolentry}{\href{https://github.com/harry-geyer/mfme}{github.com/harry-geyer/mfme}}
        %    \textbf{Mini File Manager Endpoint}
        %\end{twocolentry}
        %
        %\vspace{0.10 cm}
        %\begin{onecolentry}
        %    Lightweight file upload \& REST API (C, mongoose).
        %\end{onecolentry}
        %
        %\vspace{0.2 cm}

        \begin{twocolentry}{\href{https://github.com/harry-geyer/cv}{github.com/harry-geyer/cv}}
            \textbf{Curriculum Vitae}
        \end{twocolentry}

        \vspace{0.10 cm}
        \begin{onecolentry}
            Version-controlled CV for reproducibility (LaTeX, texlive, Make).
        \end{onecolentry}

    \section{Other}

        \begin{onecolentry}
            \begin{highlights}
                \item Languages: English (native), German (intermediate).
                \item Completed bronze, silver, and gold Duke of Edinburgh expeditions.
                \item Full and clean UK driving license since 2017.
            \end{highlights}
        \end{onecolentry}

\end{document}
